\documentclass{ximera}  
\usepackage{amsmath,amssymb}
\usepackage{amsthm}
\usepackage[position=bottom]{caption}
\theoremstyle{definition}
\usepackage{tikz}
\usepackage{wrapfig}
%\newtheorem{definition}{Definition}[section]
%\newtheorem{example}{Example}[section]
\title{The First Activity}  
\begin{document}  
\begin{abstract}  
Functions  
\end{abstract}  
\maketitle  
\section*{This is an introduction to Functions.}  
A function is a relation between sets of objects that can be thought of as a ``mathematical machine.'' This means for each input, there is exactly one output. Let’s say this explicitly.
\begin{definition}
	A \textbf{function} is a relation between sets, where for each input, there is \underline{exactly} one output.
\end{definition}
\begin{remark}
	Something as simple as a dictionary could be thought
	of as a relation, as it connects words to definitions.
	However, a dictionary is not a function, as there
	are words with multiple definitions. On the other
	hand, if each word only had a single definition, then
	it would be a function.
\end{remark}
Moreover, whenever we talk about functions, we should try to explicitly state what type of things the inputs are and what type of things the outputs are. In calculus, functions often define a relation from (a subset of) the real numbers to (a subset of) the real numbers.
\begin{example}
	Consider the function \(f\) that maps from the real numbers to
	the real numbers by taking a number and mapping it to its cube:
\begin{align*}
1&\to 1\\
-2&\to -8\\
1.5&\to 3.375
\end{align*}
and so on. This function can be described by the formula \(f(x)= x^3\) or by the plot shown in the Figure \ref{fig1}.
\end{example}
%\begin{image}
%\begin{figure*}[h]
\begin{wrapfigure}{r}{0.4\textwidth}
	\centering
	\resizebox{0.3\textwidth}{!}{
	\begin{tikzpicture}  
	\begin{axis}[  
	xmin=-2,  
	xmax=2,  
	ymin=-9,  
	ymax=9,  
	axis lines=center,  
	xlabel=$x$,  
	ylabel=$y$,  
	every axis y label/.style={at=(current axis.above origin),anchor=south},  
	every axis x label/.style={at=(current axis.right of origin),anchor=west},  
	]  
	\addplot [ultra thick, blue, smooth] {x^3};  
	\end{axis}  
	\end{tikzpicture}
	}
	\caption{\small A plot of\(f(x)=x^3\) . Here we can see that
		for each input (a value on the x-axis), there is exactly one output (a value on the y-axis).}\label{fig1}
%\end{image}
%\end{figure*}
\end{wrapfigure}
\textbf{Warning} A function is a relation (such that for each input, there is exactly one output) between sets and should not be confused with either its formula or its plot.
\begin{itemize}
	\item A formula merely describes the mapping using algebra.
	\item A plot merely describes the mapping using pictures.
\end{itemize}
\begin{exercise}\label{exercise1}  
In Figure \ref{fig2} we see a plot of \(y = f ( x )\) . What is \(f(4) = \answer{2}\)  
\end{exercise}
\begin{wrapfigure}{r}{0.3\textwidth}
	\centering
		\resizebox{0.4\textwidth}{!}{
		\begin{tikzpicture}  
		\begin{axis}[  
		xmin=-2,  
		xmax=5,  
		ymin=-2,  
		ymax=4,  
		axis lines=center,  
		xlabel=$x$,  
		ylabel=$y$,  
		every axis y label/.style={at=(current axis.above origin),anchor=south},  
		every axis x label/.style={at=(current axis.right of origin),anchor=west},  
		]  
		\addplot [ultra thick, blue, smooth] {abs(x-2)};  
		\end{axis}  
		\end{tikzpicture}
	}
		\caption{}
		\label{fig2}
	%\end{image}
	%\end{figure*}
\end{wrapfigure}
%\begin{figure}[h]
%	\centering
%	\begin{tikzpicture}
%	\draw (0,0)--(2,0)--(2,2)--(0,2)--cycle;
%	\end{tikzpicture}
%	\caption{dfsa}
%\end{figure}
%\begin{exercise}  
%  Choose the best place to work on mathematics.  
%  \begin{multipleChoice}  
%    \choice{At the library}  
%    \choice[correct]{At the caf\'e}  
%    \choice{In your office}  
%  \end{multipleChoice}  
%\end{exercise}  
%\begin{question}  
%	Which of the following functions has a graph which is a parabola?  
%	\begin{multipleChoice}  
%		\choice[correct]{$y=x^2+3x-3$}  
%		\choice{$y = \frac{1}{x+2}$}  
%		\choice{$y=3x+1$}  
%	\end{multipleChoice}  
%\end{question} out

\end{document} 