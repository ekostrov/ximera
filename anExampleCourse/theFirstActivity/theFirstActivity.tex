\documentclass{ximera}  
\usepackage{amsmath,amssymb}
\usepackage{amsthm}
\theoremstyle{definition}
%\newtheorem{definition}{Definition}[section]
%\newtheorem{example}{Example}[section]
\title{The First Activity}  
\begin{document}  
\begin{abstract}  
Functions  
\end{abstract}  
\maketitle  
\section*{This is an introduction to Functions.}  
A function is a relation between sets of objects that can be thought of as a ``mathematical machine.'' This means for each input, there is exactly one output. Let’s say this explicitly.
\begin{definition}
	A \textbf{function} is a relation between sets, where for each input, there is \underline{exactly} one output.
\end{definition}
Moreover, whenever we talk about functions, we should try to explicitly state what type of things the inputs are and what type of things the outputs are. In calculus, functions often define a relation from (a subset of) the real numbers to (a subset of) the real numbers.
\begin{remark}
Something as simple as a dictionary could be thought
of as a relation, as it connects words to definitions.
However, a dictionary is not a function, as there
are words with multiple definitions. On the other
hand, if each word only had a single definition, then
it would be a function.
\end{remark}
\begin{exercise}  
  Choose the best place to work on mathematics.  
  \begin{multipleChoice}  
    \choice{At the library}  
    \choice[correct]{At the caf\'e}  
    \choice{In your office}  
  \end{multipleChoice}  
\end{exercise}  
\begin{question}  
	Which of the following functions has a graph which is a parabola?  
	\begin{multipleChoice}  
		\choice[correct]{$y=x^2+3x-3$}  
		\choice{$y = \frac{1}{x+2}$}  
		\choice{$y=3x+1$}  
	\end{multipleChoice}  
\end{question} out

\end{document} 